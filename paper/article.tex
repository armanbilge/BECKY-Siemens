% !TEX TS-program = pdflatex
% !BIB TS-program = biber
\documentclass[12pt,letterpaper]{article}

% Layout and formatting
\usepackage[margin=1in]{geometry}
\usepackage{acronym}
\usepackage{csquotes}
\usepackage{fixltx2e}
\usepackage{mathptmx}
\usepackage{setspace}
\usepackage{units}
\frenchspacing

% Bibliography
\usepackage[american]{babel}
\usepackage[backend=biber,style=apa]{biblatex}
\DeclareLanguageMapping{american}{american-apa}
\bibliography{bibliography.bib}

% Graphics
\usepackage{graphicx}
\usepackage[pdf]{pstricks}
\psset{unit=1in,linewidth=0.02}

% Figures, tables, and captions
\usepackage{float}
\usepackage{booktabs}
\usepackage{caption}
\captionsetup{labelfont=bf,font=small,labelsep=period}
\usepackage{subcaption}
\usepackage[rightcaption]{sidecap}

% Math
\usepackage{amsmath}
\usepackage{algorithm2e}
\usepackage{mathtools}

\title{BECKY:\\Bayesian Estimation of Coevolutionary KrYteria}
%\author{Arman Bilge}

% Acronyms
\acrodef{MCMC}{Markov chain Monte Carlo}
\acrodef{GTR}{general time reversible}

\begin{document}

\begin{titlepage}
\null
\vfil
\let\newpage\relax\maketitle
\vfil
\centering
\includegraphics[width=2.5in]{figures/icon.pdf}
\vfil
\end{titlepage}

\newpage

\doublespacing

\section*{Introduction}

Considerable . To date, there are a number of tools available for .

It is important to note that the host-symbiont [reconciliation] problem is largely similar to the gene tree--species tree [reconciliation] problem. Accordingly, it is worth comparing these 

\textcite{Akerborg:2009} compare [reconciliation] methods to the evolution of methods for phylogenetic evidence, highlighting the shift to the probabilistic, and in particular, Bayesian approach.

However, problems similar to those of parsimony inference of phylogeny also plague parsimony inference of cophylogenetic mappings.

A likelihood implementation has many advantages \parencite{Charleston:2009}.

\begin{figure}
\centering
\begin{subfigure}[b]{0.2\textwidth}
\centering
\begin{pspicture}(1,1)
\psline(0,0.5)(0.5,0.5)
\psline(0.5,0.3)(1,0.3)
\psline(0.5,0.7)(1,0.7)
\psline(0.5,0.3)(0.5,0.7)
\psline[linestyle=dashed](0,0.6)(0.4,0.6)
\psline[linestyle=dashed](0.4,0.4)(1,0.4)
\psline[linestyle=dashed](0.4,0.8)(1,0.8)
\psline[linestyle=dashed](0.4,0.4)(0.4,0.8)
\end{pspicture}
\caption{cospeciation}
\end{subfigure}
\begin{subfigure}[b]{0.2\textwidth}
\centering
\begin{pspicture}(1,1)
\psline(0,0)(1,0)
\psline[linestyle=dashed](0,0.1)(1,0.1)
\psline[linestyle=dashed](0.5,0.1)(0.5,0.3)
\psline[linestyle=dashed](0.5,0.3)(1,0.3)
\end{pspicture}
\caption{duplication}
\end{subfigure}
\begin{subfigure}[b]{0.2\textwidth}
\centering
\begin{pspicture}(1,1)
\psline(0,0)(1,0)
\psline(0,0.4)(1,0.4)
\psline[linestyle=dashed](0,0.1)(1,0.1)
\psline[linestyle=dashed](0.5,0.1)(0.5,0.5)
\psline[linestyle=dashed](0.5,0.5)(1,0.5)
\end{pspicture}
\caption{host-shift}
\end{subfigure}
\begin{subfigure}[b]{0.2\textwidth}
\centering
\begin{pspicture}(1,1)
\psline(0,0)(1,0)
\psline[linestyle=dashed](0,0.1)(0.8,0.1)
\rput(0.8,0.1){\textsf{X}}
\end{pspicture}
\caption{loss}
\end{subfigure}
\caption{Diagrams of the four cophylogenetic events considered in my model, which was coined the four-event model by \textcite{Ronquist:2003}, where the host phylogeny and the symbiont dashed. All events are branch events (i.e., they can occur at any point), except cospeciation, which is a nodal event (and thus can occur only when the host speciates).}
\end{figure}

\section*{Materials and Methods}

%\begin{equation}
%P\left(M,\Theta_M|D\right) \propto \int\limits_{T_H \times T_S} \overbrace{P\left(d_H|t_H,\theta_H\right)}^{\mathclap{\text{Felsenstein likelihood for host tree}}} \underbrace{P\left(d_S|t_S,\theta_S\right)}_{\mathclap{\text{Felsenstein likelihood for symbiont tree}}} P\left(d_M|M,\theta_M,t_H,t_S\right) P\left(M\right) \textrm{d}T_H \times T_S 
%\end{equation}

\begin{equation}
P\left(m,\theta|D\right) \propto \int\limits_{T_\textrm{H} \times T_\textrm{S}}  P\left(d_\textrm{M}|m, \theta\right) P\left(t_\textrm{H}|d_\textrm{H}\right) P\left(t_\textrm{S}|d_\textrm{S}\right) P\left(\theta\right) \textrm{d} t_\textrm{H},t_\textrm{S} 
\end{equation}

This integral can be approximated via a \ac{MCMC} analysis.

Nevertheless, it is possible to model these events on the principle of observation; i.e., what specific event or events are we observing.

For example, we can consider molecular evolution.

\section*{Results}

\section*{Discussion}

\section*{Conclusions and Future Work}

The effects of the cophylogenetic likelihood on mixing of the MCMC chain needs to be seriously investigated.

\printbibliography

\end{document}
