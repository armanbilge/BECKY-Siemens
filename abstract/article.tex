% !TEX TS-program = pdflatex
% !BIB TS-program = biber
\documentclass[12pt,letterpaper]{article}

% Layout and formatting
\usepackage[margin=1in]{geometry}
\usepackage{acronym}
\usepackage{csquotes}
\usepackage{fixltx2e}
\usepackage{url}
\usepackage[T1]{fontenc}
\usepackage{mathptmx}
\usepackage{setspace}
\usepackage{units}
%\usepackage{etoolbox}
%\BeforeBeginEnvironment{equation}{\begin{singlespace}}
%\AfterEndEnvironment{equation}{\end{singlespace}\noindent\ignorespaces}
%\BeforeBeginEnvironment{align}{\begin{singlespace}}
%\AfterEndEnvironment{align}{\end{singlespace}\noindent\ignorespaces}
\frenchspacing

% Graphics
\usepackage{graphicx}
\usepackage[pdf]{pstricks}
\psset{unit=1in,linewidth=0.02,arrows=C-C,arrowsize=0.2}

% Bibliography
\usepackage[american]{babel}
\usepackage[backend=biber,style=apa]{biblatex}
\DeclareLanguageMapping{american}{american-apa}
%\bibliography{bibliography.bib}
\newcommand{\aposcite}[2]{\citeauthor{#1}'s #2 (\citeyear{#1})}
\usepackage{doi}

% Figures, tables, and captions
\usepackage{float}
\usepackage{booktabs}
\usepackage[hypcap]{caption}
\captionsetup{labelfont=bf,font=small,labelsep=period}
\usepackage{subcaption}
\usepackage[rightcaption]{sidecap}
\usepackage[plain]{fancyref}

% Math
\usepackage{amsmath}
\usepackage[algoruled]{algorithm2e}
\usepackage{mathtools}
\usepackage{commath}

\newcommand{\R}{\ensuremath{\mathcal{R}}}

\title{Bayesian Reconstruction of Coevolutionary Histories}

% Acronyms
\acrodef{GMTC}{geographic mosaic theory of coevolution}
\acrodef{GTR}{general time reversible}
\acrodef{MCMC}{Markov chain Monte Carlo}
\acrodef{ESS}{estimated sample size}
\acrodef{MCC}{maximum clade credibility}

% Reusable figures
\newcommand{\pscophylogeny}{
\begin{pspicture}(18,12)
\psset{unit=0.5cm,linewidth=0.2}
\psline[linecolor=blue](0,0)(10,10)
\psline[linecolor=blue](4,0)(2,2)
\psline[linecolor=blue](8,0)(4,4)
\psline[linecolor=blue](12,0)(14,2)
\psline[linecolor=blue](16,0)(8,8)
\psline[linecolor=red](1,0)(11,10)
\psline[linecolor=red,arrows=-o](17,0)(9,8)
\psline[linecolor=red,arrows=-o](13,0)(15,2)
\psline[linecolor=red](7,3)(14,3)
\psline[linecolor=red,arrows=<-](10,3)(14,3)
\psline[linecolor=red](10,0)(7,3)
\psline[linecolor=red,arrows=-o](9,0)(5,4)
\psline[linecolor=red,arrows=-o](4,1)(3,2)
\rput{135}(4,1){\LARGE\textcolor{red}{\textsf{\textbf{x}}}}
\psline[linecolor=red](18,0)(17,1)
\psline[linecolor=red,arrows=*-](16,1)(17,1)
\end{pspicture}
}

\begin{document}

\maketitle
\thispagestyle{empty}

\doublespacing

\section*{Abstract}

Symbiotic interactions are a fundamental aspect of life and understanding how these relationships coevolve remains an important goal in biology. Phylogenetic trees show the ancestral relationships between a group of organisms and can be used to learn about evolutionary processes. After constructing a phylogeny for a host and symbiont organism from molecular sequences, we can create a map of ancestral symbionts to ancestral hosts called a reconciliation. The reconciliation often reveals discrepancies between the two trees which we can explain with the biological events duplication, host-switch, and loss. Current approaches to this problem apply a penalizing score to each event and optimize the reconciliation between the two trees to reduce these events. I instead assign each event a rate and develop an algorithm to calculate the probability of a symbiont tree given a host tree, a reconciliation, and the event rates and implement it in a Bayesian MCMC framework where all parameters are simultaneously estimated or integrated out. I test my algorithm on data simulated under the same coevolutionary model and find good, but not perfect, reconstruction of coevolutionary histories, as well as a flaw in the operator. Finally, I discuss potential extensions for my model, particularly to coevolutionary theory.

\end{document}