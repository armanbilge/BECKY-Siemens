% !TEX TS-program = pdflatex
% !BIB TS-program = biber
\documentclass[12pt,letterpaper]{article}

% Layout and formatting
\usepackage[margin=1in]{geometry}
\usepackage{acronym}
\usepackage{csquotes}
\usepackage{fixltx2e}
\usepackage{url}
\usepackage[T1]{fontenc}
\usepackage{mathptmx}
\usepackage{setspace}
\usepackage{units}
%\usepackage{etoolbox}
%\BeforeBeginEnvironment{equation}{\begin{singlespace}}
%\AfterEndEnvironment{equation}{\end{singlespace}\noindent\ignorespaces}
%\BeforeBeginEnvironment{align}{\begin{singlespace}}
%\AfterEndEnvironment{align}{\end{singlespace}\noindent\ignorespaces}
\frenchspacing

% Graphics
\usepackage{graphicx}
\usepackage[pdf]{pstricks}
\psset{unit=1in,linewidth=0.02,arrows=C-C,arrowsize=0.2}

% Bibliography
\usepackage[american]{babel}
\usepackage[backend=biber,style=apa]{biblatex}
\DeclareLanguageMapping{american}{american-apa}
%\bibliography{bibliography.bib}
\newcommand{\aposcite}[2]{\citeauthor{#1}'s #2 (\citeyear{#1})}
\usepackage{doi}

% Figures, tables, and captions
\usepackage{float}
\usepackage{booktabs}
\usepackage[hypcap]{caption}
\captionsetup{labelfont=bf,font=small,labelsep=period}
\usepackage{subcaption}
\usepackage[rightcaption]{sidecap}
\usepackage[plain]{fancyref}

% Math
\usepackage{amsmath}
\usepackage[algoruled]{algorithm2e}
\usepackage{mathtools}
\usepackage{commath}

\newcommand{\R}{\ensuremath{\mathcal{R}}}

\title{Bayesian Reconstruction of Coevolutionary Histories}

% Acronyms
\acrodef{GMTC}{geographic mosaic theory of coevolution}
\acrodef{GTR}{general time reversible}
\acrodef{MCMC}{Markov chain Monte Carlo}
\acrodef{ESS}{estimated sample size}
\acrodef{MCC}{maximum clade credibility}

% Reusable figures
\newcommand{\pscophylogeny}{
\begin{pspicture}(18,12)
\psset{unit=0.5cm,linewidth=0.2}
\psline[linecolor=blue](0,0)(10,10)
\psline[linecolor=blue](4,0)(2,2)
\psline[linecolor=blue](8,0)(4,4)
\psline[linecolor=blue](12,0)(14,2)
\psline[linecolor=blue](16,0)(8,8)
\psline[linecolor=red](1,0)(11,10)
\psline[linecolor=red,arrows=-o](17,0)(9,8)
\psline[linecolor=red,arrows=-o](13,0)(15,2)
\psline[linecolor=red](7,3)(14,3)
\psline[linecolor=red,arrows=<-](10,3)(14,3)
\psline[linecolor=red](10,0)(7,3)
\psline[linecolor=red,arrows=-o](9,0)(5,4)
\psline[linecolor=red,arrows=-o](4,1)(3,2)
\rput{135}(4,1){\LARGE\textcolor{red}{\textsf{\textbf{x}}}}
\psline[linecolor=red](18,0)(17,1)
\psline[linecolor=red,arrows=*-](16,1)(17,1)
\end{pspicture}
}

\begin{document}

\maketitle
\thispagestyle{empty}

\doublespacing

\section*{Executive Summary}

Whether an animal, plant, bacterium, or even a virus, most organisms depend on interactions with other species. Understanding how these interactions emerge and change is an important goal in biology. This process is called coevolution because multiple interacting organisms are evolving together, generally a host organism and a symbiont organism. Similar to constructing a family tree, and observing how traits such as eye- and hair-color are passed from one generation to the next, we can construct family trees of organisms from their DNA to learn how their traits evolve over time. Unlike our knowledge of our parents and grandparents, whose eye- and hair-colors we know, we do not know what the ancestral organisms were like. One solution to this problem is to make a guess about the ancestors' traits, and try to use it to explain the traits that we see today. For organismic interactions, we can guess which ancestral host was partnered with which ancestral symbiont and try to explain their current partnerships with biological events. Examples of these events are host-switches, when an organism changes its partner. I create an algorithm to calculate the probability of a guess using these events, so that by making lots of guesses with a computer we can reconstruct the ancestral partnerships between a host and symbiont organism. I test my algorithm on simulated data and show that it can find a good, but not perfect, reconstruction of these ancestral partnerships.

\end{document}